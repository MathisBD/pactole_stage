%%  -*- ispell-local-dictionary: "american-w_accents" -*-
\documentclass[utf8x]{beamer}
\usepackage{etex}
\usepackage[english]{babel}
\usepackage[utf8x]{inputenc}
\usepackage[T1]{fontenc}
\usepackage{amsmath,amsthm,amsfonts}
\usepackage[all, color, pdf]{xy}
\usepackage{mathtools}
\usepackage{xcolor}
%\usepackage{realboxes} % for \Colorbox
\usepackage{skull} % for \skull
%\usepackage{ulem}
\usepackage{marvosym} % for \Lightning
\usepackage{amssymb}
\usepackage{cancel}
\usepackage{fourier} % for \danger
\usepackage[scaled=0.85]{beramono}
\renewcommand{\ttdefault}{pcr}
\usepackage{listings}
\usepackage{lstcoq}
\lstset{language=Coq}

%\lstset{backgroundcolor=\color{black!15}}
%\lstset{basicstyle=\scriptsize\bf\ttfamily,escapeinside={(*@}{@*)}}

\lstset{escapeinside={(*@}{@*)}}

%\usecolortheme{crane}


\setbeamertemplate{navigation symbols}{}
\setbeamercovered{invisible}

\definecolor{mygreen}{rgb}{0,.6,0}
\newcommand\hfilll{\hskip 0pt plus 1filll}
\newcommand\cit[1]{\hfill {\scriptsize \textcolor{purple}{[#1]}}}
\newcommand\gris[1]{\uncover{{\color{gray} #1}}}
\newcommand\auteur[2]{#1~\textsc{#2}}
\newcommand\mycite[1]{\textcolor{purple}{#1}}
\newcommand\Checkmark{\textcolor{mygreen}{\checkmark}}
\newcommand<>{\alertb}[1]{\alt#2{\textcolor{blue}{#1}}{#1}}

% Rising dots
\makeatletter
\def\revddots{\mathinner{\mkern1mu\raise\p@
\vbox{\kern7\p@\hbox{.}}\mkern2mu
\raise4\p@\hbox{.}\mkern2mu\raise7\p@\hbox{.}\mkern1mu}}
\makeatother


\usetikzlibrary{arrows,chains,calc}

\title{}
\author{\auteur{Mathis}{Bouverot-Dupuis}}
\institute{CNAM -- ??}
\date[12 juillet 2022]{12 juillet 2022}


\begin{document}

\frame{\maketitle}

\begin{frame}
  \tableofcontents
\end{frame}

\section{Model}
%%%%%%%%%%%%%%%

\begin{frame}
  \frametitle{Suzuji \& Yamashita's Model}

  \begin{overlayarea}{\linewidth}{10em}
  \begin{minipage}{.45 \linewidth}
    Very simple (dumb) robots :
    \begin{itemize}
      \item<1-> Points in $R^2$ (can overlap)
      \item<2-> Anonymous
      \item<3-> No direct communication
      \item<4-> No common direction/scale 
      \item<5-> Strong multiplicity detection
      \item<6-> Same robogram
    \end{itemize}
  \end{minipage}
  \hfill
  \begin{minipage}{.45 \linewidth}
    \includegraphics<1-3>[width=\linewidth, keepaspectratio]{figures/points_scattered}
    \includegraphics<4->[width=\linewidth, keepaspectratio]{figures/points_referential}
  \end{minipage}
  \end{overlayarea}
\end{frame}


% align en coq

\begin{frame}[fragile]
  \frametitle{Aligment : Goal}
  \begin{overlayarea}{\linewidth}{10em}
    Goal : move robots to a common line, and make them stay on the line.
    
    %\includegraphics<1>[width=\linewidth, keepaspectratio]{figures/points_align}
    \begin{lstlisting}[language=Coq]
      Definition round (r:robogram) (da:demonic_action) cfg
    \end{lstlisting}
  \end{overlayarea}
\end{frame}

\section{Weber Point}
%%%%%%%%%%%%%%%%%%%%%

\section{Alignment}
%%%%%%%%%%%%%%%

\section{Gathering}
%%%%%%%%%%%%%%%%%%%
% -> citer propriété de Zohir Bouzid

\end{document}